\documentclass{article}
\usepackage{graphicx}
\usepackage[utf8]{inputenc}
\usepackage{fullpage}
\usepackage{listings}
\usepackage{xcolor}
\usepackage{url}
\usepackage[linesnumbered,ruled,vlined]{algorithm2e}
\usepackage{enumitem}
\usepackage{mathrsfs}
\usepackage{amssymb}
\usepackage{hyperref}
\usepackage{mathtools}

\definecolor{mygreen}{rgb}{0,0.6,0}

\hypersetup{
    colorlinks=true,
    linkcolor=cyan,
    urlcolor=cyan}

\parindent0in
\pagestyle{plain}
\thispagestyle{plain}

\newtheorem{theorem}{Teorema}
\newtheorem{corollary}{Corolário}[theorem]
\newtheorem{lemma}[theorem]{Lema}
\newtheorem{definition}{Definição}

\newcommand{\assignment}{Abstract of statistical inference}
\newcommand{\duedate}{August 01, 2022}


% \renewcommand\thesubsection{\arabic{subsection}}

\title{Resumo de Inferência Estatística}
\author{}
\date{}

\begin{document}

Fundação Getúlio Vargas\hfill\\
Inferência Estatística\hfill\textbf{\assignment}\\
Wellington Silva\hfill\textbf{Last Update:} \duedate\\
\smallskip\hrule\bigskip

{\let\newpage\relax\maketitle}
\maketitle

\section*{Content}

\textbf{\nameref{s0}}
\vspace{2.6mm}

\textbf{\nameref{s1}}
\vspace{2.6mm}

\textbf{\nameref{s2}}
\vspace{2.6mm}

\textbf{\nameref{s3}}
\vspace{2.6mm}

\textbf{\nameref{s4}}
\vspace{2.6mm}

\textbf{\nameref{s5}}
\vspace{2.6mm}

\textbf{\nameref{s6}}
\vspace{2.6mm}

\textbf{\nameref{s7}}
\vspace{2.6mm}

\textbf{\nameref{s8}}
\vspace{2.6mm}

\textbf{\nameref{s9}}
\vspace{2.6mm}

\textbf{\nameref{s10}}
\vspace{2.6mm}

\textbf{\nameref{s11}}
\vspace{2.6mm}

\textbf{\nameref{s12}}
\vspace{2.6mm}

\textbf{\nameref{s13}}
\vspace{2.6mm}

\textbf{\nameref{s14}}
\vspace{2.6mm}

\textbf{\nameref{s15}}
\vspace{2.6mm}

\textbf{\nameref{s16}}
\vspace{2.6mm}

\textbf{\nameref{s17}}
\vspace{2.6mm}

\textbf{\nameref{s18}}
\vspace{2.6mm}

\textbf{\nameref{s19}}
\vspace{2.6mm}

\textbf{\nameref{s20}}
\vspace{2.6mm}

\textbf{\nameref{s21}}
\vspace{2.6mm}

\textbf{\nameref{s22}}

\newpage

%%%%%%%% Revisão de Probabilidade.
\section*{Aula 0: Revisão de Probabilidade}
\label{s0}

\begin{theorem}[Desigualdade de Marvok]
Seja X uma variável aleatória não-negativa e $t > 0$. Então

\begin{equation}
    P(X \geq t) \leq \frac{E[X^n]}{t^n}.
\end{equation}
\end{theorem}

\begin{theorem}[Desigualdade de Chebychev]
Seja uma variável aleatória com média $E[Y] := \mu$ e variância $Var(Y) := \sigma^2$, ambas finitas. Mas uma vez, $t > 0$. Então

\begin{equation}
    P(|Y - \mu| \geq t) \leq \frac{Var(Y)}{t^2}.
\end{equation}
\end{theorem}

\begin{theorem}[Média e variância em uma amostra i.i.d.]
Sejam $X_1, X_2, \ldots, X_n$ variáveis aleatórias i.i.d., com média $\mu$ e variância $\sigma^2$. Temos que

\begin{enumerate}
    \item[(i)] $E[\bar{X}_n] = \mu$
    \item[(ii)] $Var(\bar{X}_n) = \frac{\sigma^2}{n}$
\end{enumerate}
\end{theorem}

\begin{definition}[Convergência em probabilidade]
Dizemos que uma sequência de variáveis aleatórias converge em probabilidade para b se, $\forall \varepsilon > 0$, temos

$$\lim_{n \to \infty} P(|Z_n - b| < \varepsilon) = 1$$

Neste caso, escrevemos $Z_n \xrightarrow{p} b$.
\end{definition}

\begin{theorem}[Lei Fraca dos Grandes Números]
Sejam $X_1, X_2, \ldots, X_n$ variáveis aleatórias i.i.d., com média $\mu$ e variância $\sigma^2$. Então

$$\bar{X}_n \xrightarrow{p} \mu$$
\end{theorem}

\begin{definition}[Convergência quase certa]
Dizemos que uma sequência de variáveis aleatórias $(Z_n)_{n \leq 1}$ converge quase certamente para b se

$$P(\lim_{n \to \infty} Z_n = b) = 1$$
\end{definition}

\begin{theorem}[Lei Forte dos Grandes Números]
Sejam $X_1, X_2, \ldots, X_n$ variáveis aleatórias i.i.d., com média $\mu$. Então

$$P(\lim_{n \to \infty} \bar{X}_n = \mu) = 1$$
\end{theorem}

\begin{theorem}[Teorema Central do Limite (Lindenberg e Lévy)]
Sejam $X_1, X_2, \ldots, X_n$ variáveis aleatórias i.i.d., com média $\mu$ e variância $\sigma^2$. Então, para cada x, temos

$$\lim_{n \to \infty} P \left ( \frac{\bar{X}_n - \mu}{\sigma / \sqrt{n}} \leq x \right ) = \Phi(x),$$

onde,

$$\Phi(x) := \frac{1}{\sqrt{2 \pi}} \int_0^x exp \left(- \frac{t^2}{2} \right) d t$$

é a função de distribuição (cumulativa) normal padrão.
\end{theorem}

\section*{Aula 1: O que é e para que serve Inferência Estatística?}
\label{s1}

\begin{definition}[Modelo estatístico: informal]
Um \textbf{modelo estatístico} consiste na identificação de variáveis aleatórias de interesse (observáveis e potencialmente observáveis), na especificação de uma distribuição conjunta para as variáveis aleatórias observáveis e na identificação dos parâmetros $(\theta)$ desta distribuição conjunta. Às vezes é conveniente assumir que os parâmetros são variáveis aleatórias também, mas para isso é preciso especificar uma distribuição conjunta para $\theta$.
\end{definition}

\begin{definition}[Modelo estatístico: formal]
Seja $\mathcal{X}$ um espaço amostral qualquer, $\Theta$ um conjunto não-vazio arbitrário e $\mathcal{P(X)}$ o conjunto de todas as distribuições de probabilidade em $\mathcal{X}$. Um modelo estatístico \textbf{paramétrico} é uma função $P: \Theta \rightarrow \mathcal{P(X)}$ que associa a cada $\theta \in \Theta$ uma distribuição de probabilidade $P_\theta$ em $\mathcal{X}$.
\end{definition}

\begin{definition}[Afirmação probabilística]
Dizemos que uma afirmação é \textbf{probabilística} quando ela utiliza conceitos da teoria de probabilidade para falar de um objeto.
\end{definition}

\begin{definition}[Inferência Estatística]
Uma \textbf{inferência estatística} é uma \textbf{afirmação probabilística} sobre uma ou mais partes de um modelo estatístico.
\end{definition}

\begin{definition}[Estatística]
Suponha que temos uma coleção de variáveis aleatórias $X_1, X_2, \ldots, X_n \subseteq \mathbf{R}^n$ e uma função $r: \mathbf{X} \rightarrow R^m$. Dizemos que a variável aleatória $T = r(X_1, X_2, \ldots, X_n)$ é uma \textbf{estatística}.
\end{definition}

\begin{definition}[Permutabilidade]
Uma coleção finita de variáveis aleatórias $X_1, X_2, \ldots, X_n$ com densidade conjunta f é dita \textbf{permutável} se

\begin{equation}
f(x_1, x_2, \ldots, x_n) = f(x_{\pi(1)}, x_{\pi(2)}, \ldots, x_{\pi(n)})
\end{equation}

para qualquer permutação $\pi = \{\pi(1), \pi(2), \ldots, \pi(n)\}$ dos seus elementos. Uma coleção finita é permutável se qualquer subconjunto finito é permutável.
\end{definition}

\section*{Aula 2:  Distribuição a priori e a posteriori}
\label{s2}
\begin{definition}[Distribuição a priori]
Se tratamos o parâmetro $\theta$ como uma variável aleatória, então a \textbf{distribuição a priori} é a distribuição que damos a $\theta$ antes de observarmos as outras variáveis aleatórias de interesse. Vamos denotar a função de densidade/massa de probabilidade da priori por $\xi(\theta)$.
\end{definition}

\begin{definition}[Distribuição a posteriori]
Considere o problema estatístico com parâmetros $\theta$ e variáveis aleatórias observáveis $X_1, X_2, \ldots, X_n$. A distribuição condicional de $\theta$ dados os valores observados das variáveis aleatórias, $\textbf{x} := \{x_1, x_2, \ldots, x_n\}$ é a \textbf{distribuição a posteriori} de $\theta$, denotamos por $\xi(\theta \mid \textbf{x})$ a f.d.p./f.m.p. condicional a $X_1 = x_1, X_2 = x_2, \ldots, X_n = x_n$.
\end{definition}

\begin{theorem}[Distribuição a posteriori: derivação]
Considere a amostra aleatória $X_1, X_2, \ldots, X_n$ de uma distribuição com f.d.p./f.m.p. $f(x \mid \theta)$. Se a distribuição a priori é $\xi(\theta)$, temos

\begin{equation}
\xi(\theta \mid x) = \frac{\xi(\theta) \Pi_{i = 1}^{n} f(x_i \mid \theta)}{g_n(x)}, \ \theta \in \Omega
\end{equation}

Chamamos $g_n(x)$ de distribuição marginal de $X_1, X_2, \ldots, X_n$.
\end{theorem}

\begin{definition}[Função de verossimilhança]
Quando encaramos a f.d.p./f.m.p. $f(x_1, x_2, \ldots, x_n \mid \theta)$ como uma função do parâmetro $\theta$, chamamos esta função de \textbf{função de verossimilhança}, e podemos denotá-la como $L(\theta; x)$ ou, quando a notação não criar ambiguidade, simplesmente $L(\theta)$.
\end{definition}

\section*{Aula 3: Prioris conjugadas e função de perda}
\label{s3}
\begin{definition}[Hiper-parâmetros]
Seja $\xi(\theta \mid \phi)$ a distribuição a priori para o parâmetro $\theta$, indexada por $\phi \in \Phi$. Dizemos que $\phi$ é(são) o(s) \textbf{hiper-parâmetro(s)} da priori de $\theta$.
\end{definition}

\begin{definition}[Priori conjugada]
Suponha que $X_1, X_2, \ldots$ sejam condicionalmente independentes dado $\theta$, com f.d.p./f.m.p. $f(x\mid\theta)$. Defina

\begin{equation}
\Psi = \left \{ f: \Omega \rightarrow (0, \infty), \int_\Omega f dx = 1 \right \}
\end{equation}

onde $\Omega$ é o espaço de parâmetros. Dizemos que $\Psi$ é uma \textbf{família de distribuições conjugadas} para $f(x \mid \theta)$ se $\forall f \in \Psi$ e toda realização \textbf{x} de $X = X_1, X_2, \ldots, X_n$

\begin{equation}
\frac{f(\textbf{x} \mid \theta) f(\theta)}{\int_\Omega f(\textbf{x} \mid \theta)f(\theta) d \theta} \in \Psi
\end{equation}
\end{definition}

\begin{theorem}[Distribuição a posteriori da média de uma normal]
Suponha que $X_1, X_2, \ldots, X_n$ formam uma amostra aleatória com distribuição normal e com média desconhecida $\theta$ e variância $\sigma^2 > 0$, conhecida e fixa. Suponha que $\theta \sim Normal(\mu_0, v_0^2)$ a priori. Então

\begin{equation}
\xi (\theta \mid x, \sigma^2) = \frac{1}{\sqrt{2 \pi \sigma^2}} exp \left (\frac{(\theta - \mu_1)^2}{2v_1^2}\right ),
\end{equation}

onde

\begin{equation}
\mu_1 := \frac{\sigma^2 \mu_0 + n v_0^2 \overline{x}_n}{\sigma^2 + n v_ 0^2} \ \mathrm{e} \ v_1^2 := \frac{\sigma^2 v_0^2}{\sigma^2 + n v_0^2}
\end{equation}
\end{theorem}

\begin{definition}[Priori imprópria]
Seja $\xi : \Lambda \rightarrow (0, \infty), \Omega \subseteq \Lambda$, uma função tal que $\int_\Omega \xi(\theta) d \theta = \infty$. Se utilizamos $\xi$ como uma p.d.f.~\footnote{p.d.f. - ``probability density function'' ou função de densidade de probabilidade} para $\theta$, dizemos que $\xi$ é uma \textbf{priori imprópria} para $\theta$.
\end{definition}

\begin{definition}[Estimador]
Sejam $X_1, X_2, \ldots, X_n$ variáveis aleatórias com distribuição conjunta indexada por $\theta$. Um \textbf{estimador} de $\theta$ é qualquer função real $\delta$: $X_1, X_2, \ldots, X_n \rightarrow \mathbb{R}^d, d \geq 1$.
\end{definition}

\begin{definition}[Estimativa]
Dizemos que o valor de $\delta$ avaliado nas realizações de $X_1, X_2, \ldots, X_n$, $\textbf{x} = \{x_1, x_2, \ldots, x_n\}, \ \delta(\textbf{x})\}$ é uma \textbf{estimativa} de $\theta$.
\end{definition}

\begin{definition}[Função de perda]
Uma função de perda é uma função real em duas variáveis 
\begin{equation}
L: \Omega \times \mathbb{R}^d \rightarrow \mathbb{R},
\end{equation}

em que dizemos que o estatístico \textit{perde} $L(\theta, a)$ se o parâmetro vale $\theta$ e a estimativa dada vale a.
\end{definition}

\subsection*{Famílias Conjugadas}

Se $X_1,\ldots,X_n$ são iid e seguem a distribuição da coluna ``Dados'' na tabela~\ref{tab:1}.

\textbf{Notações}: $\bar {x}_n= \frac{1}{n}\sum_{i=1}^n x_i;~~~
y=\sum_{i=1}^nx_i$
\begin{table}[!htb]
    \centering
    \begin{tabular}{|c|c|c|}
        \hline \textbf{Dados} & \textbf{Priori} & \textbf{Posteriori}\\ \hline
         $\mathrm{Bernoulli}(\theta)$&$\mathrm{Beta}(\alpha,\beta)$ & $\mathrm{Beta}(\alpha+y,\beta+n-y)$ \\ \hline
         $\mathrm{Poisson}(\theta)$&$\mathrm{Gama}(\alpha,\beta)$&$\mathrm{Gama}(\alpha+y,\beta+n)$\\ \hline 
         $\mathrm{Normal}(\mu,\sigma^2)$ & $\mathrm{Normal}(\mu_0,v_0^2)$& $\mathrm{Normal}\left(\frac{\sigma^2\mu_0+nv_0^2\bar{x}_n}{\sigma^2+nv_0^2},\frac{\sigma^2v_0^2}{\sigma^2+nv_0^2}\right)$ \\ \hline
         $\mathrm{Exp}(\theta)$ &$\mathrm{Gama(\alpha,\beta)}$ & $\mathrm{Gama}(\alpha+n,\beta+y)$ \\ \hline

    \end{tabular}
    \caption{Famílias Conjugadas}\label{tab:1}
\end{table}

\section*{Aula 4: Estimadores de Bayes e EMV}
\label{s4}
\begin{definition}[Estimador de Bayes]
Considere a perda esperada a posteriori:

\begin{equation}
E_{\theta \mid x} [L(\theta, a)] = E[L(\theta, a) \mid x] = \int_\Omega L(\theta, a) \xi (\theta \mid x) d \theta
\end{equation}

Dizemos que $\delta^*$ é um \textbf{estimador de Bayes} se, para toda realização $X = x$,

\begin{equation}
E[L(\theta, \delta^*(x)) \mid x] = \min_{a \in A} E[L(\theta, a) \mid x].
\end{equation}

Em outras palavras, um estimador de Bayes é uma função real dos dados que minimiza a perda esperada com respeito à posteriori dos parâmetros.
\end{definition}

\begin{theorem}[$\delta^*$ sob perda quadrática]
Seja $\theta$ um parâmetro tomando valores reais. Sob perda quadrática,

\begin{equation}
\delta^*(x) = E[\theta \mid X = x] = \int_\Omega \theta \xi(\theta \mid x) d \theta
\end{equation}
\end{theorem}

\begin{theorem}[$\delta^*$ sob perda absoluta]
Suponha que a função de perda é dada por

\begin{equation}
    L(\theta, \delta^*) = | \theta - \delta^*|.
\end{equation}

Dizemos que a função de perda é \textbf{absoluta}.
Seja $\theta$ um parâmetro tomando valores na reta. Sob perda absoluta, $\delta^*(x)$ é a \textbf{mediana} a posteriori, isto é,

\begin{equation}
    \int_{\infty}^{\delta^*(x)} \xi(\theta \mid x) d \theta = \frac{1}{2}
\end{equation}
\end{theorem}

\begin{definition}[Estimador consistente]
Seja $\delta_1, \delta_2, \ldots, \delta_n$ uma sequência de estimadores de $\theta$. Se quando $n \rightarrow \infty$ a sequência convergente para $\theta$, dizemos que esta é uma \textbf{sequência consistente de estimadores}.
\end{definition}

\begin{definition}[Estimador de máxima verossimilhança]
Para cada possível vetor (de observações) x, seja $\delta(x) \in \Omega$ um valor de $\theta \in \Omega$ de modo que a função de verossimilhança, $L(\theta) \propto f(x \mid \theta)$~\footnote{$\propto$ - é um operador matemático binário que indica que o valor esquerdo é proporcional ao valor direito.}, atinge o máximo.
Dizemos que $\hat\theta = \delta(\textbf{X})$ é o \textbf{estimador de máximo verossimilhança} de $\theta$ (Fisher, 1922)\footnote{Ronald Aylmer Fisher (1890-1962), biólogo e estatístico inglês.}. Quando observamos $\textbf{X} = x$, dizemos que $\delta(x)$ é uma estimativa de $\theta$. Dito de outra forma:

\begin{equation}
    \max_{\theta \in \Omega} f(\textbf{X} \mid \theta) = f(\textbf{X} \mid \hat\theta).
\end{equation}
\end{definition}


\section*{Aula 5: EMV}
\label{s5}
\begin{theorem}[Invariância do EMV]
Considere uma função $\phi: \Omega \rightarrow \mathbb{R}$. Se $\hat{\theta}$ é um EMV para $\theta$, então $\phi(\hat{\theta})$ é um EMV para $\omega = \phi(\theta)$.
\end{theorem}

\begin{theorem}[Consistência do EMV]
Defina $l(\theta) := \log f_n(x \mid \theta)$ e assuma que $X_1, \ldots, X_n \sim f(\theta_0)$, isto é, que $\theta_0$ é o valor verdadeiro do parâmetro.
Denote $E_{\theta_0}[g] := \int_{\mathcal{X}} g(x, \theta_0) f(x \mid \theta_0)\, dx$. Suponha que

\begin{itemize}
    \item $f(x_i \mid \theta)$ tem o mesmo suporte;
    \item $\theta_0$ é o ponto inferior de $\Omega$;
    \item $I(\theta)$ é diferenciável;
    \item $\hat{\theta}_{EMV}$ é única solução de $I'(\theta) = 0$.
\end{itemize}

Então

$$\hat{\theta}_{EMV} \rightarrow \theta$$

\end{theorem}

\section*{Aula 6: Método dos momentos e suficiência}
\label{s6}
\begin{definition}[Método dos momentos]
Suponha que $X_1, \ldots, X_n$ formam uma sequência aleatória com distribuição conjunta $f_n (X_1, \ldots, X_n \mid \theta), \theta \in \Omega \subseteq \mathbb{R}^k$ e que o k-ésimo momento existe. Defina $\mu_j (\theta) = E[X_1^j \mid \theta]$ e suponha que $\mu: \Omega \rightarrow \mathbb{R}^k$ é biunívoca, de modo que sua inversa é

$$\theta = M(\mu_1(\theta), \ldots, \mu_k(\theta)).$$

Dados os momentos amostrais $m_j := \frac{1}{n} \sum_{i=1}^n X_i^j, j = 1, \ldots, k$ o \textbf{estimador de momentos} (EMM) de $\theta$ é

$$\hat{\theta}_{EMM} = M(m_1, \ldots, m_k).$$
\end{definition}

\begin{theorem}[Consistência do EMM]
Suponha que $X_1, \ldots, X_n$ formam uma amostra aleatória com distribuição conjunta  $f_n (X_1, \ldots, X_n \mid \theta), \theta \in \Omega \subseteq \mathbb{R}^k$ e que o k-ésimo momento existe. Suponha que a inversa M existe e é continua. Então o EMM é \textbf{consistente} para $\theta$.
\end{theorem}

\begin{definition}[Estatística suficiente]
Seja $X_1, \ldots, X_n$ uma amostra aleatória de uma distribuição indexada pelo parâmetro $\theta$. Seja $T = r(X_1, \ldots, X_n)$ uma estatística. Dizemos que T é uma \textbf{estatística suficiente} para $\theta$ se e somente se

\begin{equation}
    f(X_1, \ldots, X_n \mid T = t, \theta) = f(X_1, \ldots, X_n \mid T = t, \theta'), \forall \theta, \theta' \in \Omega,
\end{equation}

isto é, se a distribuição condicional da amostra dado o valor da estatística não depende de $\theta$.
\end{definition}

\begin{definition}[Aleatorização auxiliar]
 Suponha que T é suficiente para $\theta$. O processo de simular $X_1', \ldots, X_n'$ dado que $T = r(X_1, \ldots, X_n)$ de modo que 
 
 \begin{equation}
     f(X_1, \ldots, X_n \mid \theta) = f(X_1', \ldots, X_n' \mid \theta), \forall \theta \in \Omega,
 \end{equation}
 
 é chamado de \textbf{aleatorização auxiliar} (em inglês, auxiliary randomisation).
\end{definition}

\begin{theorem}[Teorema de fatorização]\label{teo:fatorizacao}
Suponha que $X_1, \ldots, X_n$ perfazem uma amostra aleatória com f.d.p./f.m.p. $f(x \mid \theta), \theta \in \Omega$. Uma estatística $T = r(X_1, \ldots, X_n)$ é suficiente para $\theta$ se, e somente se, para todo $x \in \mathcal{X}$ e $\theta \in \Omega$ existem u e v não negativos tal que

\begin{equation}
    f_n(x \mid \theta) = u(x) v[r(x), \theta].
\end{equation}
\end{theorem}

\begin{definition}[Suficiência conjunta]
Dizemos que um conjunto de estatísticas $T = \{T_1, \ldots, T_n\}$ é \textbf{suficiente} (conjuntamente) se que a distribuição condicional conjunta de $X_1, \ldots, X_n$ dado $T_1 = t_1, \ldots, T_n = t_n$ não dependentes de $\theta$.
\end{definition}

\section*{Aula 7: Suficiência conjunta e mínima, teorema de Rao-Blackwell}
\label{s7}
\begin{definition}[Estatísticas de ordem]
Seja $\textbf{X} = X_1, \ldots, X_n$ uma amostra aleatória. Dizemos que $Y_1, \ldots, Y_n$ são \textbf{estatísticas de ordem} se $Y_1$ é o menor valor de \textbf{X}, $Y_2$ é o segundo menor valor e assim sucessivamente.
\end{definition}

\begin{theorem}[Estatísticas de ordem são suficientes conjuntas]
Seja $X_1, \ldots, X_n$ uma amostra aleatória com f.d.p./f.m.p. $f(x \mid \theta)$. As estatísticas de ordem $Y_1, \ldots, Y_n$ são suficientes conjuntas para $\theta$.
\end{theorem}

\begin{definition}[Suficiência mínima]
Uma estatística \textbf{T} é dita \textbf{mínima suficiente} se \textbf{T} é suficiente e é função de qualquer outra estatística suficiente. Um vetor $\textbf{T} = \{ T_1, \ldots, T_n \}$ é dito \textbf{minimamente suficiente conjunto} se é função de qualquer outro valor de estatísticas suficientes conjuntas.
\end{definition}

\begin{theorem}[EMV e Bayes são suficientes]
Se a função de verossimilhança admite fatorização pelo Teorema~\ref{teo:fatorizacao}, os estimadores de Bayes e de máxima verossimilhança são estatísticas minimamente suficientes.
\end{theorem}

\begin{definition}[Notação conveniente]
É conveniente definir que para $g : \mathcal{X}^n \rightarrow \mathbb{R}$, escrevemos

\begin{equation}
    E_\theta [g] = \int_\mathcal{X} \cdots \int_\mathcal{X} g(\textbf{x}) f_n (\textbf{x} \mid \theta) d x_1 \cdots d x_n = \int_\mathcal{X} g(\textbf{x}) f_n (\textbf{x} \mid \theta) d \textbf{x} 
\end{equation}
\end{definition}

\begin{definition}[Erro quadrático médio]
\begin{equation}
    R(\theta, \delta) := E_\theta \left [ \{ \delta(\textbf{X}) - \theta \}^2 \right ].
\end{equation}
\end{definition}

\begin{definition}[Estimador condicionado]\label{def:est. cond.}
\begin{equation}
    \delta_0 (\textbf{T}) := E_\theta \left [ \delta(\textbf{X}) \mid \textbf{T} \right ].
\end{equation}
\end{definition}

\begin{theorem}[Teorema de Rao-Blackwell]
Seja $\delta(\textbf{X})$ um estimador, \textbf{T} uma estatística suficiente para $\theta$ e seja $\delta_0(\textbf{T})$ como na Definição~\ref{def:est. cond.}. Então vale que

$$R(\theta, \delta_0) \leq R(\theta, \delta)$$

Além disso, se $R(\theta, \delta) < \infty$ e $\delta (\textbf{X})$ não é função de $\textbf{T}$, vale a desigualdade estrita:

$$R(\theta, \delta_0) < R(\theta, \delta)$$
\end{theorem}

\section*{Aula 8: Admissibilidade e viés}
\label{s8}
\begin{definition}[Admissibilidade]
Um estimador $\delta$ é dito \textbf{inadmissível} se existe outro estimador $\delta_0$ tal que $R(\theta, \delta_0) \leq R(\theta, \delta), \forall \theta \in \Omega$ e existe $\theta' \in \Omega$ tal que $R(\theta', \delta_0) < R(\theta', \delta)$. Nesse caso, dizemos que $\delta_0$ domina $\delta$. O estimador $\delta_0$ é \textbf{admissível} se (e somente se) não há nenhum estimador que o domine.
\end{definition}

\begin{definition}[Estimador não-viesado]
Um estimador $\delta(\textbf{X})$ de uma função $g(\theta)$ é dito \textbf{não-viesado} se $E_\theta[\delta(\textbf{X})] = g(\theta), \ \forall \theta \in \Omega$. Um estimador que não atende a essa condição é dito viesado. E o \textbf{víes} de $\delta$ é definido como $B_\delta(\theta) := E_\theta[\delta(\textbf{X})] - g(\theta)$.
\end{definition}

\begin{theorem}[Estimador não-viesado da variância]
Seja $\textbf{X} = \{ X_1, \ldots, X_n \}$ uma amostra aleatória, com $E[X_1] = m$ e $Var(X_1) = v < \infty$. Então

$$\delta_1(\textbf{X}) = \frac{1}{n - 1}\sum_{i = 1}^n (X_i - \overline{X}_n)^2$$

é um estimador não-viesado de v.
\end{theorem}

\section*{Aula 9: Eficiência}
\label{s9}
\begin{definition}[Informação de Fisher]\label{def:fis}
Seja X uma variável aleatória com f.d.p./f.m.p. $f(x \mid \theta), \ \theta \in \Omega \subseteq \mathbb{R}$. Suponha que $f(x \mid \theta)$ é duas vezes diferenciável com respeito a $\theta$. Defina $\lambda (x \mid \theta) = \log f(x \mid \theta)$ e

\begin{equation}
    \lambda' (x \mid \theta) = \frac{\partial \lambda (x \mid \theta)}{\partial \theta} \quad \mathrm{e} \quad \lambda'' (x \mid \theta) = \frac{\partial^2 \lambda(x \mid \theta)}{\partial \theta^2}
\end{equation}

Definimos a \textbf{informação de Fisher} como

\begin{equation}
    I(\theta) = E_\theta \left [ \{ \lambda'(x \mid \theta) \}^2 \right ] \stackrel{\mathrm{(1)}}{=} -E_\theta\left[\lambda^{\prime\prime}(x \mid \theta)\right] = Var_\theta\left(\lambda^{\prime}(x\mid \theta) \right).
\end{equation}
\end{definition}

\begin{theorem}[Informação de Fisher em uma amostra aleatória]
Seja $\textbf{X} = \{ X_1, \ldots, X_n \}$ uma amostra aleatória e seja $I_n = E_\theta [- \lambda''_n (\textbf{X} \mid \theta)]$ a \textbf{informação de Fisher} da amostra. Então

$$I_n(\theta) = n I(\theta)$$
\end{theorem}

\begin{theorem}[Teorema de Cramér-Rao]
Seja $\textbf{X} = \{ X_1, \ldots, X_n \}$ uma amostra aleatória, onde f.d.p./f.m.p. tem as mesmas premissas da Definição~\ref{def:fis}. Supondo que $T = r(\textbf{X})$ é uma estatística com variância finita. Seja $m(\theta) = E_\theta(T)$ uma função diferenciável de $\theta$. Então,

\begin{equation}
    Var_\theta(T) \geq \frac{[m'(\theta)]^2}{n I(\theta)},
\end{equation}

com igualdade apenas se existem u e v tal que

$$T = u(\theta) \lambda'_n (\textbf{X} \mid \theta) + v(\theta).$$
\end{theorem}

\begin{definition}[Estimador eficiente]
Um estimador $\delta(\textbf{X})$ é dito \textbf{eficiente} de (sua esperança) $m(\theta)$ se

$$Var_\theta(\delta) = \frac{[m'(\theta)]^2}{n I(\theta)}.$$
\end{definition}

\section*{Aula 10: Distribuição de uma estatística amostral e qui-quadrado}
\label{s10}
\begin{definition}[Distribuição qui-quadrado]
Dizemos que uma variável aleatória Y tem distribuição \textbf{qui-quadrado} com m graus de liberdade quando

\begin{equation}
    f_Y(y) = \frac{1}{2^{m/2} \Gamma(m/2)} y^{m/2 - 1} e^{-y/2}, y > 0
\end{equation}

Vemos que Y tem função geradora de momentos:

$$\psi (t) = \left ( \frac{1}{1 - 2t} \right )^{m/2}, t < 1/2.$$
\end{definition}

\begin{theorem}[Soma de variáveis aleatórias qui-quadrado]
Se $X_1, \ldots, X_n$ são variáveis aleatórias independentes com graus de liberdade $m_i$, então $W = \sum_{i = 1}^n X_i$ tem distribuição qui-quadrado com graus de liberdade $m = \sum_{i = 1}^n m_i$.
\end{theorem}

\begin{theorem}[Distribuição do quadrado de uma variável aleatória Normal padrão]
Se 

$$X \sim Normal(0, 1), Y = X^2$$ 

então, tem distribuição qui-quadrado com $m = 1$.
\end{theorem}

\section*{Aula 11: Distribuição da média e variância amostrais}
\label{s11}
\begin{theorem}[Independência da média e variância amostrais na Normal]
Seja $X_1, \ldots, X_n$ uma amostra aleatória de uma distribuição Normal com parâmetros $\mu$ e $\sigma^2$, $\overline{X}_n$ e a variância amostral $\overline{S}_n^2$, são independentes. Ademais, $\overline{X}_n \sim Normal \left ( \mu, \sigma^2 \right )$ e $\overline{S}_n^2 \sim Gama \left ( \frac{n - 1}{2}, \frac{n}{2 n^2} \right )$
\end{theorem}

\section*{Aula 12: Distribuição t de Student  e intervalos de confiança}
\label{s12}
\begin{definition}[A distribuição t de Student]
Tome, $Y \sim \mathrm{Qui-quadrado}(m)$ e $Z \sim \mathrm{Normal}(0, 1)$ e defina a variável aleatória

$$X = \frac{Z}{\sqrt{\frac{Y}{m}}}.$$

Dizemos que X tem distribuição \textbf{t de Student com m graus de liberdade}. E sabemos que

$$f_X = \frac{\Gamma(\frac{m + 1}{2})}{\sqrt{m \pi} \Gamma(\frac{m}{2})} \left ( 1 + \frac{x^2}{m} \right )^{-\frac{m + 1}{2}}, \quad x \in (-\infty, + \infty).$$
\end{definition}

\begin{theorem}[Distribuição amostral do estimador não-viesado da variância]

Considere o estimador

$$\hat{\sigma}' = \sqrt{\frac{\Delta^2}{n - 1}},$$

onde $\Delta^2 = \sum_{i = 1}^n (X_i - \overline{X}_n)^2$. Então, vale que

$$\frac{\sqrt{n} (\overline{X}_n - \mu)}{\hat{\sigma}'} \sim \mathrm{Student}(n - 1)$$
\end{theorem}

\begin{theorem}[Intervalo de confiança]
Seja $\textbf{X} = \{ X_1, \ldots, X_n \}$ uma amostra aleatória, onde cada uma tem p.d.f. $f(x \mid \theta)$, e considere uma função real $g(\theta)$. Sejam $A(\textbf{X})$ e $B(\textbf{X})$ duas estatísticas de modo de valha

\begin{equation}
    P(A(\textbf{X}) < g(\theta) < B(\textbf{X})) \geq \gamma.
\end{equation}

Dizemos que $I(\textbf{X}) = (A(\textbf{X},B(\textbf{X}))$ é um \textbf{intervalo de confiança} de $100 \gamma \%$ para $g(\theta)$. Se a desigualdade for uma igualdade para todo $\theta \in \Omega$, dizemos que o intervalo é \textbf{exato}.
\end{theorem}

\section*{Aula 13: Intervalos de confiança e Quantidades Pivotais}
\label{s13}
\begin{definition}[Intervalo de confiança unilateral]
Seja $\textbf{X} = \{ X_1. \ldots, X_n \}$ uma amostra aleatória, onde cada uma tem p.d.f. $f(x \mid \theta)$, e considere uma função real $g(\theta)$. Seja $A( \textbf{X})$ uma estatística que

$$P(A(\textbf{X}) < g(\theta)) \geq \gamma, \quad \forall \theta \in \Omega$$

dizemos que o intervalo aleatório $( A(\textbf{X}), \infty)$ é chamado de intervalo de confiança \textbf{unilateral} de $100\gamma \%$ para $g(\theta)$ (ou ainda, de intervalo de confiança \textbf{inferior} de $100\gamma \%$ para $g(\theta)$). O intervalo $(-\infty, B(\textbf{X}))$, com

$$P(g(\theta) < B(\textbf{X})) \geq \gamma, \quad \forall \theta \in \Omega$$

é definido de forma análoga, e é chamado de intervalo de confiança \textbf{superior} de $100\gamma \%$ para $g(\theta)$. Se a desigualdade é uma igualdade para todo $\theta \in \Omega$, os intervalos são chamados \textbf{exatos}.
\end{definition}

\begin{definition}[Quantidade pivotal]
Seja $\textbf{X} = \{ X_1, \ldots, X_n \}$ uma amostra aleatória com p.d.f. $f(x \mid \theta)$. Seja $V(\textbf{X}, \theta)$ uma variável aleatória cuja distribuição é a mesma para todo $\theta \in \Omega$. Dizemos que $V(\textbf{X}, \theta)$ é uma \textbf{quantidade pivotal}.
\end{definition}

\begin{theorem}[Intervalo de confiança unilateral]
Seja $\textbf{X} = \{ X_1, \ldots, X_n \}$ uma amostra aleatória com p.d.f. $f(x \mid \theta)$. Suponha que existe uma quantidade pivotal V, com c.d.f.~\footnote{c.d.f. - cumulative distribution function} continua G. Assuma que existe $r(v, \textbf{x})$ estritamente crescente em v para todo $\textbf{x}$. Finalmente, tome $0 < \gamma < 1$ e $\gamma_1 < \gamma_2$ de modo que $\gamma_2 - \gamma_1 = \gamma$. Então as estatísticas

$$A(\textbf{X}) = r(G^{-1}(\gamma_1), \textbf{X}),$$

$$B(\textbf{X}) = r(G^{-1}(\gamma_2), \textbf{X}),$$

são os limites de um intervalo de confiança de $100\gamma \%$ para $g(\theta)$.
\end{theorem}


\section*{Aula 14: Testes de hipótese I}
\label{s14}
\begin{definition}[Hipótese nula e hipótese alternativa]
Considere o espaço de parâmetros $\Omega$ e defina $\Omega_0, \Omega_1 \subset \Omega$ de modo que $\Omega_0 \cup \Omega_1 = \Omega$ e $\Omega_0 \cap \Omega_1 = \emptyset$. Definimos

$$H_0 := \theta \in \Omega_0,$$
$$H_1 := \theta \in \Omega_1,$$

E dizemos que $H_0$ é a \textbf{hipótese nula} e $H_1$ é a \textbf{hipótese alternativa}.
Se $\theta \in \Omega_1$, então dizemos que rejeitamos a hipótese nula. Por outro lado, se $\theta \in \Omega_0$, então dizemos que não rejeitamos ou falhamos em rejeitar $H_0$.
\end{definition}

\begin{definition}[Hipótese simples e hipótese composta]
Dizemos que uma hipótese $H_i$, é \textbf{simples}, se $\Omega_i = \{ \theta_i \}$, isto é, se a partição correspondente é um único ponto. Uma hipótese é dita \textbf{composta} se não é simples.
\end{definition}

\begin{definition}[Hipótese unilateral e hipótese bilateral]
Uma hipótese da forma $H_0 : \theta \leq \theta_0$ ou $H_0 : \theta \geq \theta_0$ é dita \textbf{unilateral} (``one-sided''), enquanto hipóteses da forma $H_0 : \theta \neq \theta_0$ são ditas \textbf{bilaterais} (``two-sided'').
\end{definition}

\section*{Aula 15: Testes de hipótese II}
\label{s15}
\begin{definition}[Região crítica]
O conjunto

$$S_1 := \{ \textbf{x} : | \overline{X}_n - \mu_0 | \geq c \}$$

é chamado de \textbf{região crítica} do teste.
\end{definition}

\begin{definition}[Região de rejeição]
Se $R \subseteq \mathbb{R}$ é tal que ``rejeitamos $H_0$ se $T \in R$'', então R é chamada uma \textbf{região de rejeição} para a estatística T e o teste associado.
\end{definition}

\begin{definition}[Função poder]
Seja $\delta$ um procedimento de aceitação/rejeição como visto anteriormente. A \textbf{função poder} é definida como

\begin{equation}
    \pi(\theta \mid \delta) := P(\textbf{X} \in S_1 \mid \theta) = P(T \in R \mid \theta), \ \theta \in \Omega
\end{equation}
\end{definition}

\begin{definition}[Tipos de erros]
Tipos de erros que podem ser cometidos

$$
\begin{tabular}{c|c}
   Nome & Erro cometido\\
   \hline
   Erro tipo I & Rejeitar $H_0$ quando ela é~\textbf{verdadeira}.\\
   Erro tipo II & Falhar em rejeitar $H_0$ quando ela é~\textbf{falsa}.\\
\end{tabular}
$$
\end{definition}

\begin{definition}[Tamanho/nível de um teste]
Dizemos que um teste, $\delta$, tem \textbf{tamanho} ou \textbf{nível de significância} $\alpha(\delta)$, com 

$$\alpha(\delta) := \mathrm{sup}_{\theta \in \Omega_0} \pi (\theta \mid \delta).$$
\end{definition}

\section*{Aula 16: Testes de hipótese III}
\label{s16}
\begin{definition}[O p-valor]
Para cada t, seja $\delta_t$ o teste que rejeita $H_0$ se $T \geq t$. Então, quando $T = t$, o \textbf{p-valor} vale

\begin{equation}
    p(t) := \sup_{\theta \in \Omega_0} \pi (\theta \mid \delta_t) = \sup_{\theta \in \Omega_0} P (T \geq t \mid \theta)
\end{equation}

ou seja, o \textbf{p-valor} é o tamanho do teste $\delta_t$.
\end{definition}

\section*{Aula 17: Testes e conjuntos de confiança}
\label{s17}
\begin{definition}[Intervalos de confiança e testes são equivalentes]

Suponha que dispomos de dados $\textbf{X} = \{ X_1, \ldots, X_n \}$ com f.d.p. comum $f(x \mid \theta)$, e estamos interessados em testar as hipóteses:

$$H_0 : g(\theta) = g_0,$$
$$H_1 : g(\theta) \neq g_0,$$

de modo que existe um teste $\delta_{g_0}$ com nível $\alpha_0$ destas hipóteses. Para cada $\textbf{X} = \textbf{x}$, defina

$$w(\textbf{x}) = \left\{g_0: \delta_{g_0} \:n\tilde{a}o\:rejeita\: H_0 \:dado\:que\: \textbf{X} = \textbf{x} \right\}.$$

Fazendo o nível de confiança do intervalo $\gamma = 1 - \alpha_0$, temos

$$P(g(\theta_0) \in w(\textbf{X}) \mid \theta = \theta_0) \geq \gamma, \ \forall \theta_0 \in \Omega.$$
\end{definition}

\begin{definition}[Conjunto de confiança]
Se um conjunto aleatório $w(\textbf{X})$ satisfaz

$$P(g(\theta_0) \in w(\textbf{X}) \mid \theta = \theta_0) \geq \gamma,$$

para todo $\theta_0 \in \Omega$, então chamamos $w(\textbf{X})$ de um \textbf{conjunto de confiança} para $g(\theta)$.
\end{definition}

\begin{theorem}[Testando hipóteses a partir de conjuntos de confiança]
Suponha que dispomos de dados $\textbf{X} = \{ X_1, \ldots. X_n \}$ com f.d.p. comum $f(x \mid \theta)$ e que $w(\textbf{X})$ é um conjunto de confiança para uma função de interesse $g(\theta)$. Então para todo valor $g_0$ assumido por $g(\theta)$ existe um teste $\delta_{g_0}$, de nível $\alpha_0$ que rejeita $H_0 : g(\theta) = g_0$ se e somente se $g(\theta_0) = g_0 \notin w(\textbf{X})$.
\end{theorem}

\begin{theorem}[Teste de razão de verossimilhanças (para o Teo. de Wilks)]
A estatística

$$\wedge(\textbf{x}) = \frac{\sup_{\theta \in \Omega_0 f_n (\textbf{x} \mid \theta)}}{\sup_{\theta \in \Omega f_n (\textbf{x} \mid \theta)}}$$

é chamada um \textbf{estatística de razão de verossimilhanças}. Um \textbf{teste de razão de verossimilhanças}, $\delta_k$, é um teste que rejeita $H_0$ se $\wedge(\textbf{x}) \leq k$ para uma constante k.
\end{theorem}

\begin{theorem}[Teorema de Wilks]
Suponha que temos um espaço de parâmetros com k coordenadas, $\theta = (\theta_1, \ldots, \theta_n)$ e desejamos testar a hipótese (simples) da forma

$$H_0 : \theta_j = \theta_0^j, \ j = 1, \ldots, k,$$
$$H_1 : \theta_j \neq \theta_0^j, \ j = 1, \ldots, k.$$

Então, sob condições de regularidade, temos que, à medida que $n \rightarrow \infty$,

$$-2 \log \wedge(\textbf{x}) \stackrel{\mathrm{d}}{\rightarrow} X^2 (k)$$
\end{theorem}

\section*{Aula 18: Teste t I}\label{s18}
\begin{definition}[Teste não viesado]
Suponha que desejamos testar a hipótese

$$H_0 : \theta \in \Omega_0,$$
$$H_1 : \theta \in \Omega_1,$$

através do teste $\delta$. Dizemos que $\delta$ é \textbf{não-viesado} se (e somente se) para $\theta \in \Omega_0$ e $\theta' \in \Omega_1$, vale

$$\pi (\theta \mid \delta) \leq \pi (\theta' \mid \delta),$$

ou seja, se a função poder é pelo menos tão grande no espaço onde $H_0$ é falsa ($\Omega_1$) quando no espaço em que $H_0$ é verdadeira ($\Omega_0$).
\end{definition}

\begin{definition}[Teste t]
Um teste $\delta_c$ que rejeita $H_0$ se $U \geq c$ (equiv. $U \leq c$), com $c = T^{-1}(1 - \alpha_0; n - 1)$ é chamado de um \textbf{teste t} (unicaudal) de tamanho $\alpha_0$.
\end{definition}

\begin{theorem}[Propriedades do teste t]
Suponha que $\delta_c$ rejeita $H_0$ se $U \geq c$. Então

\begin{itemize}
    \item $\mu = \mu_0 \Longrightarrow \pi(\mu, \sigma^2 \mid \delta_c) = \alpha_0$
    
    \item $\mu < \mu_0 \Longrightarrow \pi(\mu, \sigma^2 \mid \delta_c) < \alpha_0$
    
    \item $\mu > \mu_0 \Longrightarrow \pi(\mu, \sigma^2 \mid \delta_c) > \alpha_0$
    
    \item $\lim_{\mu \rightarrow -\infty} \pi(\mu, \sigma^2 \mid \delta_c) = 0$
    
    \item $\lim_{\mu \rightarrow +\infty} \pi(\mu, \sigma^2 \mid \delta_c) = 1$
    
    \item $\delta_c$ é não-viesado e tem tamanho $\alpha_0$.
\end{itemize}
\end{theorem}

\begin{theorem}[P-valor para um teste t unicaudal]
Suponha que observarmos $U = u$ e seja $T(\cdot. n - 1)$ a f.d.a. de uma distribuição t de Student com n - 1 graus de liberdade. Para a hipótese

$$H_0 : \mu \geq \mu_0,$$
$$H_1 : \mu < \mu_0,$$

o p-valor vale $T(u; n-1)$, enquanto para a hipótese

$$H_0 : \mu \leq \mu_0,$$
$$H_1 : \mu > \mu_0,$$

o p-valor vale $1 - T(u; n-1)$.
\end{theorem}

\section*{Aula 19: Teste t II}\label{s19}
\begin{theorem}[Teste pareado]
Sejam amostras \textbf{X} e \textbf{Y} (antes e depois), tais que $X_i \sim \mathrm{Normal}(\mu_1, \sigma^2)$ e $Y_i \sim \mathrm{Normal}(\mu_2, \sigma^2)$, a hipótese

$$H_0 : \mu_1 \leq \mu_2$$
$$H_1 : \mu_1 > \mu_2$$

Pode ser modelada com a variável $Z_i = X_i - Y_i$ ($Z_i \sim \mathrm{Normal}(\mu_Z = \mu_1 - \mu_2, 2 \sigma^2)$), então podemos testar hipóteses sobre $\mu_Z$ a partir de \textbf{Z}

$$H_0 : \mu_Z \leq 0$$
$$H_1 : \mu_Z > 0$$
\end{theorem}

\begin{theorem}[Teste t para duas amostras]
Considere $\textbf{X} = \{ X_1, \ldots, X_m \}$ e $\textbf{Y} = \{ Y_1, \ldots, Y_n \}$, queremos estudar a diferença das médias. Modelando em distribuição normal $X_i \sim \mathrm{Normal}(\mu_1, \sigma_1^2), \ i = 1, \ldots, m$ e $Y_j \sim \mathrm{Normal}(\mu_2, \sigma_2^2), \ j = 1, \ldots, n$. Sob a premissa de homogeneidade $\sigma_1^2 = \sigma_2^2 = \sigma^2$, podemos testar a hipótese

$$H_0 : \mu_1 \leq \mu2$$
$$H_1 : \mu_1 > \mu_2$$

computando a estatística

$$U = \frac{\sqrt{m + n - 2} (\overline{X}_m - \overline{Y}_n)}{\sqrt{(\frac{1}{m} + \frac{1}{n})(S_X^2 + S_Y^2)}}$$

onde $\overline{X}_m$ e $\overline{X}_m$ são as médias e $S_X^2$ e $S_X^2$ são a soma das variâncias.
\end{theorem}

\begin{theorem}[Relaxando a premissa de homogeneidade]
Do teorema acima, podemos relaxar a premissa de igualdade das variâncias assumindo que $\sigma_2^2 = k \sigma_1^2$, então a estatística teste vale

$$U = \frac{\sqrt{m + n - 2} (\overline{X}_m - \overline{Y}_n)}{\sqrt{(\frac{1}{m} + \frac{k}{n})(S_X^2 + \frac{S_Y^2}{n})}}$$
\end{theorem}

\section*{Aula 20: Teste f}\label{s20}
\begin{definition}[A distribuição F]
Sejam $Y \sim \mathrm{Qui-quadrado}(m)$ e $W \sim \mathrm{Qui-quadrado}(n)$. Então

$$X = \frac{Y / m}{W / n},$$

tem distribuição F com m e n graus de liberdade, com f.d.p.

$$f_X (x) = \frac{\Gamma(\frac{m + n}{2}) m^{m/2} n^{n/2}}{\Gamma(\frac{n}{2}) \Gamma(\frac{m}{2})} \cdot \frac{x^{m / 2 - 1}}{(mx + n)^{(m + n)/2}}, \ x > 0,$$
\end{definition}

\begin{theorem}[Propriedades da distribuição F] Propriedades para a distribuição F:

\begin{enumerate}
    \item Se $X \sim F(m, n)$, então $\frac{1}{X} \sim F(m, n)$;
    
    \item Se $Y \sim \mathrm{Student}(n)$, então $Y^2 \sim F(1, n)$.
\end{enumerate}
\end{theorem}

\begin{theorem}[Igualdade de duas variâncias]
Suponha $X_i \sim \mathrm{Normal}(\mu_1, \sigma_1^2), \ i = 1, \ldots, m$ e $Y_j \sim \mathrm{Normal}(\mu_2, \sigma_2^2), \ j = 1, \ldots, n$. Queremos testar

$$H_0 : \sigma_1^2 \leq \sigma_2^2$$
$$H_1 : \sigma_1^2 > \sigma_2^2$$

Para isso, vamos computar a estatística de teste

$$V = \frac{S_X^2 / (m - 1)}{S_Y^2 / (n - 1)}$$

onde $S_X^2 = \sum_{i = 1}^m (X_i - \overline{X}_m)^2$ e $S_Y^2 = \sum_{j = 1}^m (Y_j - \overline{Y}_m)^2$
\end{theorem}

\begin{definition}[O teste F]
O teste F de homogeneidade (igualdade de variâncias) é o teste $\delta_c$ que rejeita $H_0$ de $V \geq c$, para uma constante positiva c.
\end{definition}

\begin{theorem}[A distribuição de V]
Seja $V = \frac{S_X^2 / (m - 1)}{S_Y^2 / (n - 1)}$, então:

$$\frac{\sigma_2^2}{\sigma_1^2} V \sim F(m - 1, n - 1).$$

Além disso, se $\sigma_1^2 = \sigma_2^2$, $V \sim F(m - 1, n - 1)$.
\end{theorem}

\section*{Aula 21: Regressão Linear I}\label{s21}
\begin{theorem}[A linha de mínimos quadrados]
Sejam $(x_1, y_1), \ldots, (x_n, y_n)$ uma coleção de n pontos. Suponha que estamos interessados na reta

\begin{equation}
    y_i = \beta_0 + \beta_1 x_i.
\end{equation}

Os valores dos coeficientes que minimizam a  soma de quadrados são

$$\hat{\beta}_0 = \overline{y} - \hat{\beta}_1 \overline{x},$$

$$\hat{\beta}_1 = \frac{\sum_{i = 1}^n (y_i - \overline{y}) (x_i - \overline{x})}{\sum_{i = 1}^n (x_i - \overline{x})^2}.$$
\end{theorem}

\begin{definition}[Modelo linear]
Podemos construir um modelo estatístico explícito para a relação entre as variáveis X e Y:

\begin{equation}
    E[Y \mid X = x_1, \ldots, x_P] = \beta_0 + \beta_1 x_1 + \ldots + \beta_P x_P
\end{equation}

Podemos então idealizar o seguinte modelo

$$Y_i = \beta_0 \sum_{j = 1}^P \beta_j x_{ij} + \varepsilon_i, \ \varepsilon \sim \mathrm{Normal}(0, \sigma^2).$$
\end{definition}

\section*{Aula 22: Regressão Linear II}\label{s22}
\begin{theorem}[EMV para os coeficientes de uma regressão linear (simples)]
Sob as premissas já listadas, os estimadores de máxima verossimilhança para $\theta = (\beta_0, \beta_1, \sigma^2)$ são

$$\hat{\beta}_{0 EMV} = \overline{y} - \hat{\beta}_{1 EMV} \overline{x},$$

$$\hat{\beta}_{1 EMV} = \frac{\sum_{i = 1}^n (y_i - \overline{y})(x_i - \overline{x})}{\sum_{i = 1}^n (x_i - \overline{x})^2},$$

$$\hat{\sigma^2}_{EMV} = \frac{1}{n} \sum_{i = 1}^n \left ( y_i - (\hat{\beta}_{0 EMV} + \hat{\beta}_{1 EMV} x_i) \right )^2,$$

ou seja, os estimadores de máxima verossimilhança dos coeficientes minimizam a soma de quadrados da reta estimada.
\end{theorem}

\begin{theorem}[Distribuição amostral dos estimadores dos coeficientes]
$$\hat{\beta}_{0 EMV} \sim \mathrm{Normal} \left ( \beta_0, \sigma^2 \left ( \frac{1}{n} + \frac{\overline{x}^2}{s_X^2} \right ) \right ),$$

$$\hat{\beta}_{1 EMV} \sim \mathrm{Normal} \left ( \beta_0, \frac{\sigma^ 2}{s_X^2} \right ),$$

$$\mathrm{Cov} (\hat{\beta}_{0 EMV}, \hat{\beta}_{1 EMV}) = - \frac{\overline{x} \sigma^2}{s_X^2},$$

onde $s_x = \sqrt{\sum_{i = 1}^n (x_i - \overline{x})^2}.$
\end{theorem}

\begin{theorem}[Intervalos de confiança para os coeficientes de uma regressão linear]
Podemos computar intervalos de confiança para os coeficientes da regressão linear
de maneira muito similar ao que já vimos para o caso da média da Normal

$$\hat{\beta_0} \pm \hat{\sigma}' c\sqrt{\frac{1}{n} + \frac{\bar{x}^2}{s_x^2}}\quad \mathrm{e}\quad \hat{\beta_1} \pm c\frac{\hat{\sigma}'}{s_x},$$

$$\hat{\beta_0} + \hat{\beta_1}x_{pred} \pm c \hat{\sigma}' \sqrt{\frac{1}{n} + \frac{\left(x_{pred}-\bar{x}\right)^2}{s_x^2} }$$

onde $c = T^{-1}(1-\frac{\alpha_0}{2}; n-2)$ e 

$$\hat{\sigma}' := \sqrt{\frac{\sum_{i=1}^n \left(Y_i - \hat{\beta_0} - \hat{\beta_1}x_i \right)^2}{n-2}}.
$$
\end{theorem}

\begin{definition}[Testes de hipóteses para o coeficiente angular]
Em geral, estamos interessados em testar a hipótese
$$H_0 : \beta_1 = \beta^\star,$$
$$H_1 : \beta_1 \neq \beta^\star.$$

Para tanto, podemos computar a estatística

\begin{equation}
 U_1 = s_x \frac{\hat{\beta_1}-\beta^\star}{\hat{\sigma}'},
\end{equation}

e computar o p-valor como 

\begin{equation}
 P(U_1 \geq |u_1|) + P(U_1 \leq -|u_1|).
\end{equation}
Notando que $U_1$ tem distribuição t de Student com $n-2$ graus de liberdade sob $H_0$, podemos computar o p-valor exatamente.

Resultados bem similares valem para testar hipóteses sobre $\beta_0$ ou $\hat{Y}$.
\end{definition}

\begin{theorem}[Predição pontual]
Suponha que queremos prever o valor de Y para um certo $x_{pred}$ que não foi observado no experimento. Podemos compor nossa predição (pontual) como

\begin{equation}
    \hat{Y} = \hat{\beta}_0 + \hat{\beta}_1 x_{pred}.
\end{equation}

Onde a predição tem erro quadrático médio (EQM) igual a

$$E \left [ (\hat{Y} - Y)^2 \right ] = \sigma^2 \left ( 1 + \frac{1}{n} + \frac{(x_{pred} - \overline{x})^2}{s_X^2} \right ).$$
\end{theorem}

\begin{theorem}[Intervalos de \textbf{predição} para $\hat{Y}$]
A probabilidade de $\hat{Y} = \hat{\beta}_0 + \hat{\beta}_1 x_{pred}$ estar no intervalo

$$\hat{Y} \pm T^{-1} (1 - \frac{\alpha_0}{2}; n - 2) \hat{\sigma}' \sqrt{\left [ 1 + \frac{1}{n} + \frac{(x_{pred} - \overline{x})^2}{s_X^2} \right ]},$$

é $1 - \alpha_0$.
\end{theorem}

\end{document}
