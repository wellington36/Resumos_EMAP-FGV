\documentclass{article}
\usepackage{graphicx}
\usepackage[utf8]{inputenc}
\usepackage{fullpage}
\usepackage{listings}
\usepackage{xcolor}
\usepackage{url}
\usepackage[linesnumbered,ruled,vlined]{algorithm2e}
\usepackage{enumitem}
\usepackage{mathrsfs}
\usepackage{amssymb}
\usepackage{hyperref}

\definecolor{mygreen}{rgb}{0,0.6,0}

\hypersetup{
    colorlinks=true,
    linkcolor=cyan,
    urlcolor=cyan}

\parindent0in
\pagestyle{plain}
\thispagestyle{plain}

\newtheorem{theorem}{Teorema}
\newtheorem{corollary}{Corolário}[theorem]
\newtheorem{lemma}[theorem]{Lema}
\newtheorem{definition}{Definição}

\newcommand{\assignment}{Abstract of surves and surfaces}
\newcommand{\duedate}{February 19, 2021}


% \renewcommand\thesubsection{\arabic{subsection}}

\title{Resumo de Curvas e Superfícies}
\author{}
\date{}

\begin{document}

Fundação Getúlio Vargas\hfill\\
Curvas e Superfícies\hfill\textbf{\assignment}\\
Wellington Silva\hfill\textbf{Due:} \duedate\\
\smallskip\hrule\bigskip

{\let\newpage\relax\maketitle}
\maketitle

\section*{Sumário}

\textbf{\nameref{s1}}
\vspace{4.0mm}

\newpage

\section*{Aula 2}
\label{s1}

\begin{definition}
Uma \textbf{curva parametrizada} em $\mathbb{R}^n$ é uma aplicação $\gamma: I \rightarrow \mathbb{R}^n$ sendo $I \subset \mathbb{R}$ aberto.
\end{definition}

\begin{definition}
O conjunto imagem de $\gamma$, $\gamma(I) \subset \mathbb{R}^n$ é dito o \textbf{traço} de $\gamma$.
\end{definition}

\begin{definition}[Vetor tangente]
Seja $\gamma: I \rightarrow \mathbb{R}^n$ com $\gamma(t) = (\gamma_1 (t), \ldots, \gamma_n (t))$ com $\gamma_i (y)$ diferenciáveis $\forall i, i = 1 \ldots n$, o vetor

$$\gamma'(t) = (\gamma'_1 (t), \ldots, \gamma'_n (t))$$

é chamado \textbf{vetor tangente de $\gamma$} em t
\end{definition}

\begin{definition}[Curvas regulares]
Seja $\gamma (t): I \rightarrow \mathbb{R}^n$ uma curva parametrizada diferenciável. Diz-se que \textbf{$\gamma$ é regular}, quando $\gamma'(t) \neq 0,\ \forall t \in I$. 
\end{definition}

\begin{definition}[Reta tangente]
Seja $\gamma$ uma curva regular, então a \textbf{reta tangente de $\gamma$} no ponto $t_0 \in I$ é aquela que contém o ponto $\gamma(t)$ e é paralela ao vetor $\gamma'(t)$, ou seja

$$r(\lambda) = \gamma(t_0) + \lambda \gamma'(t_0)$$
\end{definition}

\begin{definition}[Comprimento de arco]
O \textbf{comprimento de arco de $\alpha$}, de $\alpha(a)$ até $\alpha(b)$ definido por $L_a^b (\alpha)$ é  

$$L_a^b (\alpha) = \int_a^b \| \alpha'(t) \| d t$$
\end{definition}

\begin{definition}
Se $\gamma: (a, b) \rightarrow \mathbb{R}^n$ é uma c.p.\footnote{curva parametrizada}, sua \textbf{velocidade no ponto} $\gamma(t)$ é $\| \gamma'(t) \|$, e a curva é dita com \textbf{velocidade unitária} se $\| \gamma'(t) \| = 1,\ \forall t \in (a, b)$ e é parametrizada por comprimento de arco.
\end{definition}

\begin{theorem}
Toda \textbf{curva regular} pode ser reparametrizada por \textbf{comprimento de arco}.
\end{theorem}

\end{document}
