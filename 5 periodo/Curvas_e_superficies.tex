\documentclass{article}
\usepackage{graphicx}
\usepackage[utf8]{inputenc}
\usepackage{fullpage}
\usepackage{listings}
\usepackage{xcolor}
\usepackage{url}
\usepackage[linesnumbered,ruled,vlined]{algorithm2e}
\usepackage{enumitem}
\usepackage{mathrsfs}
\usepackage{amsmath}
\usepackage{amssymb}
\usepackage{hyperref}

\definecolor{mygreen}{rgb}{0,0.6,0}

\hypersetup{
    colorlinks=true,
    linkcolor=cyan,
    urlcolor=cyan}

\parindent0in
\pagestyle{plain}
\thispagestyle{plain}

\newtheorem{prop}{Propriedade}
\newtheorem{theorem}{Teorema}
\newtheorem{corollary}{Corolário}[theorem]
\newtheorem{lemma}[theorem]{Lema}
\newtheorem{definition}{Definição}

\newcommand{\assignment}{Abstract of surves and surfaces}
\newcommand{\duedate}{May 25, 2021}


% \renewcommand\thesubsection{\arabic{subsection}}

\title{Resumo de Curvas e Superfícies}
\author{}
\date{}

\begin{document}

Fundação Getúlio Vargas\hfill\\
Curvas e Superfícies\hfill\textbf{\assignment}\\
Wellington Silva\hfill\textbf{Last update:} \duedate\\
\smallskip\hrule\bigskip

{\let\newpage\relax\maketitle}
\maketitle

\section*{Sumário}

\textbf{\nameref{s1}}
\vspace{4.0mm}

\textbf{\nameref{s2}}
\vspace{4.0mm}

\textbf{\nameref{s3}}
\vspace{4.0mm}

\textbf{\nameref{s4}}
\vspace{4.0mm}

\textbf{\nameref{s5}}
\vspace{4.0mm}

\textbf{\nameref{s6}}
\vspace{4.0mm}

\textbf{\nameref{s7}}
\vspace{4.0mm}

\textbf{\nameref{s8}}
\vspace{4.0mm}

\textbf{\nameref{s9}}
\vspace{4.0mm}

\textbf{\nameref{s10}}
\vspace{4.0mm}

\textbf{\nameref{s11}}
\vspace{4.0mm}

\newpage

\section*{Curvas, Reta tangente e Comprimento de arco}
\label{s1}

\begin{definition}
Uma \textbf{curva parametrizada} em $\mathbb{R}^n$ é uma aplicação $\gamma: I \rightarrow \mathbb{R}^n$ sendo $I \subset \mathbb{R}$ aberto.
\end{definition}

\begin{definition}
O conjunto imagem de $\gamma$, $\gamma(I) \subset \mathbb{R}^n$ é dito o \textbf{traço} de $\gamma$.
\end{definition}

\begin{definition}[Vetor tangente]
Seja $\gamma: I \rightarrow \mathbb{R}^n$ com $\gamma(t) = (\gamma_1 (t), \ldots, \gamma_n (t))$ com $\gamma_i (y)$ diferenciáveis $\forall i, i = 1 \ldots n$, o vetor

$$\gamma'(t) = (\gamma'_1 (t), \ldots, \gamma'_n (t))$$

é chamado \textbf{vetor tangente de $\gamma$} em t
\end{definition}

\begin{definition}[Curvas regulares]
Seja $\gamma (t): I \rightarrow \mathbb{R}^n$ uma curva parametrizada diferenciável. Diz-se que \textbf{$\gamma$ é regular}, quando $\gamma'(t) \neq 0,\ \forall t \in I$. 
\end{definition}

\begin{definition}[Reta tangente]
Seja $\gamma$ uma curva regular, então a \textbf{reta tangente de $\gamma$} no ponto $t_0 \in I$ é aquela que contém o ponto $\gamma(t)$ e é paralela ao vetor $\gamma'(t)$, ou seja

$$r(\lambda) = \gamma(t_0) + \lambda \gamma'(t_0)$$
\end{definition}

\begin{definition}[Comprimento de arco]
O \textbf{comprimento de arco de $\alpha$}, de $\alpha(a)$ até $\alpha(b)$ definido por $L_a^b (\alpha)$ é  

$$L_a^b (\alpha) = \int_a^b \| \alpha'(t) \| d t$$
\end{definition}

\begin{definition}
Se $\gamma: (a, b) \rightarrow \mathbb{R}^n$ é uma c.p.\footnote{curva parametrizada}, sua \textbf{velocidade no ponto} $\gamma(t)$ é $\| \gamma'(t) \|$, e a curva é dita com \textbf{velocidade unitária} se $\| \gamma'(t) \| = 1,\ \forall t \in (a, b)$ e é parametrizada por comprimento de arco.
\end{definition}

\begin{theorem}
Toda \textbf{curva regular} pode ser reparametrizada por \textbf{comprimento de arco}.
\end{theorem}

\section*{Difeomorfismo e Reparametrização}
\label{s2}

\begin{definition}[Difeomorfismo]
Dado os conjuntos abertos $U \subset \mathbb{R}^n$ e $V \subset \mathbb{R}^n$. Uma bijeção $f: U \rightarrow V$ é dita \textbf{difeomorfismo} quando f e $f^{-1}$ são diferenciáveis.
\end{definition}

\begin{definition}[Reparametrização]
A curva $\beta(s)$ é dita uma \textbf{reparametrização} de $\alpha(t): I \subset \mathbb{R} \rightarrow \mathbb{R}^2$ regular quando dados $I_0 \subset \mathbb{R}$ e $\phi: I_0 \rightarrow I$ difeomorfismo. Temos $\beta(S) = \alpha(\phi(S)))$.
\end{definition}

\begin{definition}
Seja $\alpha(t): (a, b) \rightarrow \mathbb{R}^2$ r $\beta(S): (c, d) \rightarrow \mathbb{R}^2$. Então

\begin{itemize}
    \item $\beta(S)$ é uma reparametrização positiva de $\alpha$ se $\phi'(S) > 0,\ \forall S$
    \item $\beta(S)$ é uma reparametrização negativa de $\alpha$ se $\phi'(S) < 0,\ \forall S$
\end{itemize}
\end{definition}

\begin{definition}
Qualquer reparametrização de uma c.p. regular é regular (i.e. difeomorfismos preservam regularidade).
\end{definition}

\begin{prop}
A função L (comprimento de arco) é um difeomorfismo.
\end{prop}

\begin{definition}
Toda curva regular $\alpha: I \rightarrow \mathbb{R}^2$ admite reparametrização por comprimento de arco.
\end{definition}

\section*{Ângulo e Curvatura}
\label{s3}

\begin{definition}[Função Ângulo]
Dada uma curva diferenciável $\gamma: I \rightarrow S^1$, onde $S^1$ é o círculo de $\mathbb{R}^2$ com centro na origem e raio 1, diz-se que $\theta: I \rightarrow \mathbb{R}$ é uma \textbf{função-ângulo} de $\gamma$, quando

$$\gamma(s) = (cos(\theta(s)), sen(\theta(s)),\ \forall s \in I$$
\end{definition}

\begin{definition}[Curvatura]
Seja $\alpha: I \rightarrow \mathbb{R}$ unit-speed. Designando-se o vetor tangente de $\alpha$ em $s \in I$ por $T(s)$, podemos afirmar que a curva $T(s) = I \rightarrow S^1$ admite função ângulo

$$T(s) = (cos(\theta(s)), sen(\theta(s)),\ \forall s \in I$$

Daí a \textbf{curvatura} de $\alpha$ em $s \in I$ é definida por

$$K(s) = \theta'(s) = det(\alpha'(s), \alpha''(s))$$
\end{definition}

\section*{Teorema Fundamental das Curvas Planas}
\label{s4}

\begin{theorem}[Função-ângulo diferenciável]
Seja $\gamma: I \rightarrow S^1$ uma curva diferenciável. Então, $\gamma$ admite uma função ângulo $\theta: I \rightarrow \mathbb{R}$, a qual é diferenciável. Além disso, toda função-ângulo de $\gamma$, a qual é diferenciável, difere de $\theta$ por uma constante.
\end{theorem}

\begin{corollary}
Seja $\alpha : I \subset \mathbb{R} \rightarrow \mathbb{R}$ e seja $\beta(s) = \alpha(\theta(s))$ a parametrização por comprimento de arco de $\alpha$, a curvatura de $\alpha$ em $t \in I$ é $K_\alpha(t)$, e, por definição é a curvatura de $\beta$ em $\theta^{-1}(t)$, isto é

$$K_\alpha := K_\beta(\theta^{-1}(t))$$
\end{corollary}

\begin{definition}[Diedro de Frenet]
Seja $\alpha: I \subset \mathbb{R} \rightarrow \mathbb{R}^2$ uma curva regular parametrizada por comprimento de arco. Dado $s \in I$, o vetor $N(s) = JT(s)$ é dito o vetor normal de $\alpha$ em $s \in I$. A base ortonormal de $\mathbb{R}^2$ formado por $T(s)$ e $N(s)$ é chamada \textbf{Dietro de Frenet} em s.
\end{definition}

\begin{definition}[Movimento Rígido]
$\Phi: \mathbb{R}^2 \rightarrow \mathbb{R}^2$ é dita \textbf{movimento rígido}, quando preserva distancia, isto é, para quaisquer $p, q \in \mathbb{R}^2$

$$\| \Phi(p) - \Phi(q) \| = \| p - q \|$$
\end{definition}

\begin{theorem}
Seja $\Phi: A + p_0$ um movimento rígido direto de $\mathbb{R}^2$ e $\alpha: I \rightarrow \mathbb{R}^2$ uma curva regular parametrizada por comprimento de arco. Então, $\beta = \Phi \circ \alpha: I \rightarrow \mathbb{R}^2$ é uma curva regular de $\mathbb{R}^2$, parametrizada por comprimento de arco, tal que

$$K_\alpha(s) = K_\beta(s) \ \forall s \in I$$
\end{theorem}

\begin{theorem}[Teorema Fundamental da Teoria Local das Curvas Planas]
Sejam I um intervalo aberto da reta e $K: I \rightarrow \mathbb{R}$ uma função diferenciável.

\begin{enumerate}
    \item Então existe uma curva diferenciavel $\alpha: I \rightarrow \mathbb{R}^2$, unit-speed, cuja função curvatura coincide com K.
    
    \item Além disso, para toda $\beta: I \rightarrow \mathbb{R}^2$, unit-speed, que cumpre $K_\beta = K$, existe um movimento rígido $\Phi: \mathbb{R}^2 \rightarrow \mathbb{R}^2$ tal que $\alpha = \Phi \circ \beta$
\end{enumerate}
\end{theorem}

\section*{Curvas Regulares no $\mathbb{R}^3$}
\label{s5}

\begin{definition}[Curvas no $\mathbb{R}^3$]
As curvas diferenciáveis no $\mathbb{R}^3$, são definidas de forma análoga ao $\mathbb{R}^2$, isto é, uma \textbf{curva no $\mathbb{R}^3$} é uma aplicação diferenciável de um intervalo I (aberto) em $\mathbb{R}^3$, da forma

$$\alpha(t) = (x(t), y(t), z(t)),\ t \in I$$

Onde x, y e z são diferenciáveis, e a curva é dita \textbf{regular} quando

$$\alpha'(t) = (x'(t), y'(t), z'(t)) \neq (0, 0, 0),\ t \in I$$
\end{definition}

\begin{prop} De forma análoga vale para $\mathbb{R}^3$ que
\begin{itemize}
    \item Comprimento de arco é invariável por reparametrização.
    
    \item Toda curva regular admite reparametrização \textit{unit-speed} $(\| \alpha'(t) \| = 1)$.
\end{itemize}
\end{prop}

\begin{definition}[Curvatura no $\mathbb{R}^3$]
Dada uma curva $\alpha: I \rightarrow \mathbb{R}$ regular parametrizada por comprimento de arco, a \textbf{curvatura} de $\alpha$ em $s \in I$ é definida como

$$K_\alpha(s) = \| \alpha''(s) \|$$
\end{definition}

\begin{definition}[2-regular]
Seja uma curva regular $\alpha: I \rightarrow \mathbb{R}^3$ \textit{unit-speed}, e $K_\alpha(s) > 0, \forall s$, ou seja, $\alpha''(s) \neq 0,\ \forall s$. Então dizemos que $\alpha$ é \textbf{2-regular}.
\end{definition}

\begin{definition}[Triedro de Frenet]
Para $\alpha$ 2-regular, seja $T(s) = \alpha'(s)$ (vetor tangente), $N(s) = \frac{\alpha''(s)}{\| \alpha''(s) \|}$ (vetor normal) e $B(s) = T(s) \times N(s)$ (vetor binormal). Desse modo estabelecemos um referencial chamado \textbf{Triedro de Frenet} formado pelos vetores $\{ T(s), N(s), B(s) \}$, onde,

$$ \left \{
\begin{array}{lll}
    B(s) = T(s) \times N(s) \\
    N(s) = B(s) \times T(s) \\
    T(s) = N(s) \times B(s)
\end{array} \right .
$$
\end{definition}

\begin{definition}[Curvatura e Torção]
Seja uma curva $\alpha$ 2-regular em $\mathbb{R}^3$ não necessariamente parametrizada por comprimento de arco, então a \textbf{curvatura e a torção} de $\alpha$ são definidas respectivamente como

$$K_\alpha(t) = \frac{\| \alpha''(t) \times \alpha'(t) \|}{\| \alpha'(t) \|^3}$$

$$\mathcal{T}(t) = \frac{\langle (\alpha'(t) \times \alpha''(t)), \alpha'''(t) \rangle}{\| \alpha'(t) \times \alpha''(t) \|^2}$$
\end{definition}

\begin{theorem}
Seja $\alpha: I \rightarrow \mathbb{R}^3$ uma curva 2-regular \textit{unit-speed}, então

$$\alpha \textit{ é plana} \Longleftrightarrow \mathcal{T}(s) \equiv 0,\ \forall s \in I$$
\end{theorem}

\section*{Teorema Fundamental das Curvas Espaciais}
\label{s6}

\begin{theorem}[Teorema Fundamental da Teoria Local das Curvas Espaciais]
Sejam I um intervalo aberto, $K: I \rightarrow \mathbb{R}$ uma função positiva diferenciável e $\mathcal{T}: I \rightarrow \mathbb{R}$ uma função diferenciável

\begin{enumerate}
    \item Então existe uma curva diferenciavel $\alpha: I \rightarrow \mathbb{R}^3$, unit-speed, tal que K e $\mathcal{T}$ concedem com a curvatura e torção de $\alpha$ respectivamente
    
    \item Além disso, $\forall \beta: I \rightarrow \mathbb{R}^3$, unit-speed, que cumpre $K_\beta = K$ e $\mathcal{T}_\beta = \mathcal{T}$ existe um movimento rígido $\Phi: \mathbb{R}^3 \rightarrow \mathbb{R}^3$ tal que $\alpha(s) = \Phi(\beta(s))$ 
\end{enumerate}
\end{theorem}

\section*{Superfícies Regulares}
\label{s7}

\begin{definition}[Superfícies Regulares]
Um conjunto $S \subset \mathbb{R}^3$ é dito uma \textbf{superfície regular}, quando é localmente difeomorfo a $\mathbb{R}^2$. Mais precisamente, quando, $\forall p \in S$, exite um difeomorfismo

$$X: U \subset \mathbb{R}^2 \rightarrow V \subset S$$

onde U é um aberto de $\mathbb{R}^2$ e V é um aberto relativo de S. A aplicação X é dita, então uma parametrização local de S em p.
\end{definition}

\begin{definition}
Sendo o difeomorfismo de uma superfície da forma

$$X(u, v) = (x(u, v), y(u, v), z(u, v)),\ (u,v) \in V$$

definimos as \textbf{derivadas parciais} de X como sendo

$$X_u(u, v) = \left( \frac{\partial x}{\partial u}(u, v), \frac{\partial y}{\partial u}(u, v), \frac{\partial z}{\partial u}(u, v) \right)$$

$$X_v(u, v) = \left( \frac{\partial x}{\partial v}(u, v), \frac{\partial y}{\partial v}(u, v), \frac{\partial z}{\partial v}(u, v) \right)$$

e se $X_u$ e $X_v$ são L.I. então produzem um plano tangente no ponto p.
\end{definition}

\begin{prop}
Se S é uma superfície regular temos que:

\begin{enumerate}[label=(\alph*)]
    \item A aplicação $X(u, v) = (x(u, v), y(u, v), z(u, v))$ é diferenciável de $C^\infty$ quando x, y e z tem derivadas parciais de todas as ordens.
    
    \item Para todo $q: (u, v) \in U$, a diferencial de X em q, $dX_q: \mathbb{R}^2 \rightarrow \mathbb{R}^3$ é injetiva, nesse caso, garante-se a existência do plano tangente $T_p S$.
\end{enumerate}
\end{prop}

\begin{theorem}[Função Inversa]
Seja F diferenciável e $p \in A$ tal que $dF_p$ é injetora. Então existe uma vizinhança $U \subset A$ de p. Tal que $F(U)$ é aberto em $\mathbb{R}^n$ e a restrição $F_U$ é um difeomorfismo de U sobre $F(U)$.
\end{theorem}

\section*{Superfícies e Atlas}
\label{s8}
\begin{definition}[Definição de Superfície]
O subconjunto S de $\mathbb{R}^3$ é uma \textbf{superfície} se $\forall p \in S$, existe um aberto U em $\mathbb{R}^2$ e um aberto W em $\mathbb{R}^3$ contendo p tal que $S \cap W$ é \textbf{homeomorfo} a U.
\end{definition}

\begin{definition}[Atlas]
Uma coleção de parametrizações que cobrem uma superfície S é dita \textbf{atlas de S} e cada uma das parametrizações é dita uma \textbf{carta}.
\end{definition}

\begin{definition}[Curvas regulares enquanto subconjuntos]
Diz-se que um subconjunto $C \subset \mathbb{R}^3$ é uma \textbf{curva regular}, quando para cada $p \in C$, existe um intervalo aberto $I \subset \mathbb{R}$ e um difeomorfismo $\alpha: I \rightarrow \alpha(I) \subset C$ em que $\alpha(I)$ é um aberto relativo de C.
\end{definition}

\begin{definition}[Valor Regular]
Dados um aberto $O \subset \mathbb{R}^3$ e uma função diferenciável $\varphi: O \rightarrow \mathbb{R}$. Dizemos que $q \in \mathbb{R}$ é \textbf{valor regular} de $\varphi$ quando $\forall p \in \varphi^-1(\{q\}) \subset O$ a derivada

$$d\varphi_p: \mathbb{R}^3 \rightarrow \mathbb{R}$$

é não nula, isto é, $\nabla \varphi(p) \neq 0$
\end{definition}

\begin{prop}
A \textbf{imagem inversa} de um valor regular de uma função diferenciável definida em um aberto do $\mathbb{R}^3$, quando não vazia, é uma superfície regular.
\end{prop}

\section*{Espaços topológicos $\mathbb{R}^n$ Parte I}
\label{s9}
\begin{definition}[Bola aberta]
Dado $a \in \mathbb{R}^n$ e um número real $r > 0$, a \textbf{bola aberta} de centro a e raio r em $\mathbb{R}^n$ é o conjunto

$$B(a, r) = \{ x \in \mathbb{R}^n\ |\ \| x - a \| \leq r \}$$

e respectivamente definimos \textbf{bola fechada} como

$$B(a, r) = \{ x \in \mathbb{R}^n\ |\ \| x - a \| \geq r \}$$
\end{definition}

\begin{definition}[Conjunto Limitado]
Um conjunto é dito \textbf{limitado} quando existe uma bola que o contém, ou seja,

$$\exists a \in \mathbb{R}^n\ \mathrm{e}\ r>0\ \mathrm{t.q.}\ X \subset B(a, r)$$
\end{definition}

\begin{definition}[Aplicação limitada]
Dado um conjunto A uma \textbf{aplicação} $f: A \rightarrow \mathbb{R}^n$ é dita limitada quando seu conjunto imagem é limitada.
\end{definition}

\begin{definition}[Conjunto aberto]
Um conjunto $A \subset \mathbb{R}^n$ é dito \textbf{aberto} quando $\forall a \in A \exists r > 0$ tal que $B(a, r) \subset A$ (e a é dito ponto interior de A).
\end{definition}

\begin{definition}[Aplicação aberta]
Diz-se que uma aplicação $f: \mathbb{R}^n \rightarrow \mathbb{R}^m$ é \textbf{aberta} quando $\forall A \subset \mathbb{R}^n\ \mathrm{aberto}, f(A) \subset \mathbb{R}^m$ é aberto.
\end{definition}

\begin{prop}[Propriedades dos abertos]
Propriedades fundamentais dos conjuntos abertos

\begin{enumerate}
    \item O conjunto vazio e o espaço $\mathbb{R}^n$ são abertos.
    
    \item A \textbf{intersecção} de uma família finita de abertos é aberta.
    
    \item A \textbf{união} de uma família qualquer de abertos é aberta.
\end{enumerate}
\end{prop}

\begin{definition}[Espaço topológico]
Um espaço topológico é um par $(X, T)$ em que X é um conjunto e T é uma família de subconjuntos de X, chamados abertos, que satisfazem as propriedades acima. Diz-se, então, que a família T define uma topologia
\end{definition}

\begin{theorem}
Uma sequência $(X_k)$ em $\mathbb{R}^n$ converge para $a \in \mathbb{R}^n$ se, e somente se, $\forall r > 0,\ \exists k_0 \in \mathbb{N}$ tal que se $k \geq k_0$ então $x_k \in B(a, r)$.
\end{theorem}

\begin{definition}[Conjunto fechado]
Um conjunto $F \subset \mathbb{R}^n$ é dito \textbf{fechado} quando seu complementar é aberto.
\end{definition}

\begin{prop}[Propriedades dos fechados]
Propriedades fundamentais dos conjuntos fechados

\begin{enumerate}
    \item O conjunto vazio e o espaço $\mathbb{R}^n$ são fechados.
    
    \item A \textbf{intersecção} de uma família qualquer de fechados é um conjunto fechado.
    
    \item A \textbf{união} de uma família finita de fechados é fechado.
\end{enumerate}
\end{prop}

\begin{definition}[Aplicação fechada]
Diz-se que uma aplicação $f: \mathbb{R}^n \rightarrow \mathbb{R}^m$ é \textbf{fechada} quando leva fechados de $\mathbb{R}^n$ em fechados de $\mathbb{R}^m$.
\end{definition}

\begin{definition}[Aderência]
Um ponto $a \in \mathbb{R}^n$ é dito \textbf{aderente} a um conjunto $X \subset \mathbb{R}^n$ se existe uma sequência de pontos de X que convergem para a.
\end{definition}

\begin{definition}[Fecho]
O \textbf{fecho} de X, denotado por $\overline{X}$, é o conjunto formado por todos os pontos de $\mathbb{R}^n$ que são aderentes a X.
\end{definition}

\begin{theorem}
$F \subset \mathbb{R}^n$ é fechado $\Longleftrightarrow \overline{F} = F$.
\end{theorem}

\begin{definition}[Bordo]
A fronteira (ou bordo) de um conjunto $X \subset \mathbb{R}^n$ é o conjunto $\partial X = \overline{X} \cap \overline{\mathbb{R}^n - X}$.
\end{definition}

\section*{Espaços topológicos $\mathbb{R}^n$ Parte II}
\label{s10}
\begin{definition}[Aberto relativo]
Sejam X subconjunto de $\mathbb{R}^n$ e $A \subset X$. Diz-se que A é \textbf{aberto relativo} a X ou \textbf{aberto relativamente} à X quando existe um aberto $U \subset \mathbb{R}^n$ tal que $A = U \cap C$.
\end{definition}

\begin{definition}[Cisão]
Uma \textbf{cisão} de um conjunto $X \subset \mathbb{R}^n$ é uma decomposição do mesmo em dois conjuntos disjuntos que são ambos, abertos em X, isto é, $A, B \subset \mathbb{R}^n$ tais que

\begin{itemize}
    \item $X = A \cup B$
    
    \item $A \cap B = \O$
    
    \item A e B abertos em X
\end{itemize}
\end{definition}

\begin{definition}[Conexidade]
Um conjunto $X \subset \mathbb{R}^n$ é dito \textbf{conexo} se a única cisão que admite é a trivial ($X = X \cup \O$) caso contrario é dito desconexo.
\end{definition}

\begin{definition}[Homeomorfismo]
Diz-se que dois espaços $(X_1, T_1)$ e $(X_2, T_2)$. São \textbf{homeomorfos} quando existe bijeção $\varphi: X_1 \rightarrow X_2$ tal que para quaisquer abertos $A_1 \in T_1$ e $A_2 \in T_2$ tem-se que $\varphi(A_1) \in T_2$ e $\varphi^{-1}(A_2) = T_1$. Logo $\varphi$ é dito \textbf{homeomorfismo}.
\end{definition}

\begin{definition}[Continuidade]
Dados $X, Y \subset \mathbb{R}^n$, $f: X \rightarrow Y$ é \textbf{contínua} em $a \in X$ se $\forall \varepsilon > 0,\ \exists \delta > 0$ tal que $x \in X$ e $\| x - a \| < \delta \Rightarrow \| f(x) - f(a) \| < \varepsilon$.
\end{definition}

\begin{theorem}
Dados $X \subset \mathbb{R}^n$ e $Y \subset \mathbb{R}^m$ uma bijeção $f: X \rightarrow Y$ é um homeomorfismo se e só se f e $f^{-1}$ são continuas.
\end{theorem}

\begin{definition}[Isomorfismo]
É uma aplicação que preserva uma estrutura e pode ser revertida com uma aplicação inversa.
\end{definition}

\section*{Primeira forma fundamental}
\label{s11}
\begin{theorem}[Teorema da função inversa]
Seja $F: U \subset \mathbb{R}^n \rightarrow \mathbb{R}^m$ diferenciável e $dF_p : \mathbb{R}^n \rightarrow \mathbb{R}^m$ isomorfismo. Então, existem abertos $V \subset U$ e $W \subset F(U)$, tais que se $p \in V$, então $F(p) \in W$ e $F|_v : V \rightarrow W$ é um difeomorfismo.
\end{theorem}

\begin{definition}[Vetor tangente]
Dado um ponto p de uma superfície regular S, diz-se que $w \in \mathbb{R}^n$ é um \textbf{vetor tangente} a S em p, se existe uma curva $\alpha: (- \varepsilon, \varepsilon) \rightarrow S,\ \varepsilon > 0$, tal que, $\alpha(0) = p$ e $\alpha'(0) = w$.
\end{definition}

\begin{theorem}[Primeira forma fundamental]
Seja S uma superfície regular e $p \in S$. A \textbf{primeira forma fundamental} de S em p

$$I_p : T_p S \rightarrow \mathbb{R}$$

é a forma quadrática associada à restrição do produto interno canônico de $\mathbb{R}^3$ ao plano tangente de S em p, $T_p S$, isto é

$$I_p(w) = \langle w,w \rangle^2 = \| w \|^2,\ w \in T_p S$$

Dada uma parametrização $X: U \subset \mathbb{R}^2 \rightarrow X(U) \subset S$ de S, as funções


\begin{equation*}
    E(u, v) = \langle X_u(u, v), X_u(u, v) \rangle
\end{equation*}

\begin{equation*}
    F(u, v) = \langle X_u(u, v), X_v(u, v) \rangle
\end{equation*}

\begin{equation*}
    G(u, v) = \langle X_v(u, v), X_v(u, v) \rangle
\end{equation*}

São os coeficientes da primeira forma fundamental de S relativos a X, isto é, a matriz de $I_{X(u, v)}$ com respeito a base $\{X_u, X_v\}$, de $T_{X(u, v)} S$

$$
\begin{bmatrix}
E & F\\
F & G
\end{bmatrix}
$$

E $\forall w = a X_u(u, v) + b X_v(u, v) \in T_{X_(u, v)} S$ tem-se

$$I_{X_(u, v)}(w) = a^2 E(u, v) + 2 a b F(u, v) + b^2 G(u, v)$$
\end{theorem}

\end{document}
