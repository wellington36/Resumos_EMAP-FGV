\documentclass{article}
\usepackage{graphicx}
\usepackage[utf8]{inputenc}
\usepackage{fullpage}
\usepackage{listings}
\usepackage{xcolor}
\usepackage{url}
\usepackage[linesnumbered,ruled,vlined]{algorithm2e}
\usepackage{enumitem}
\usepackage{mathrsfs}
\usepackage{amssymb}
\usepackage{hyperref}
\usepackage{amsmath}

\definecolor{mygreen}{rgb}{0,0.6,0}

\hypersetup{
    colorlinks=true,
    linkcolor=cyan,
    urlcolor=cyan}

\parindent0in
\pagestyle{plain}
\thispagestyle{plain}

\newtheorem{prop}{Proposition}
\newtheorem{theorem}{Teorema}
\newtheorem{corollary}{Corolário}[theorem]
\newtheorem{lemma}[theorem]{Lema}
\newtheorem{definition}{Definição}

\newcommand{\assignment}{Abstract of calculus in a complex variable}
\newcommand{\duedate}{March 2, 2022}


% \renewcommand\thesubsection{\arabic{subsection}}

\title{Resumo de Cálculo em uma Variável Complexa}
\author{}
\date{}

\begin{document}

Fundação Getúlio Vargas\hfill\\
Cálculo em uma Variável Complexa\hfill\textbf{\assignment}\\
Wellington Silva\hfill\textbf{Due:} \duedate\\
\smallskip\hrule\bigskip

{\let\newpage\relax\maketitle}
\maketitle

\section*{Sumário}

\textbf{\nameref{s1}}
\vspace{4.0mm}

\textbf{\nameref{s2}}
\vspace{4.0mm}

%\textbf{\nameref{s3}}
%\vspace{4.0mm}

\newpage

\section*{Números Complexos e propriedades (Week 1)}
\label{s1}

\begin{prop} As seguintes propriedades valem para quaisquer $z, w, t \in \mathbb{C}$:

\begin{enumerate}[label=(\alph*)]
    \item $z + (w + t) = (z + w) + t$
    \item $z + w = w + z$
    \item $0 + z = z$
    \item $z + (-z) = 0$
    \item $z \cdot (w \cdot t) = (z \cdot w) \cdot t$
    \item $zw = wz$
    \item $1 \cdot z = z$
    \item $z \cdot z^{-1} = 1$ se $z \neq 0$
    \item $z \cdot (w + t) = z \cdot w + z \cdot t$
\end{enumerate}
\end{prop}

\begin{definition}
Um número complexo z é da forma $z = x + iy, \ x,y \in \mathbb{R}$ e $i = \sqrt{-1}$, que podemos escrever como um par de variáveis de $\mathbb{R}^2$ de forma que $z = (x, y)$.
\end{definition}

\begin{definition}[Soma e produto nos complexos]
Seja $z = (x, y)$ e $w = (a, b),\ x,y,a,b, \in \mathbb{R}$, definimos soma e produto, para manter consistência com as propriedades acima, da seguinte forma

\begin{align}
    &z + w = (x + a, y + b) \nonumber \\
    &z \cdot w = (xa - yb, xb + ya) \nonumber
\end{align}
\end{definition}

\begin{definition}[O Módulo]
Seja $z = x + iy$ um complexo, então o \textbf{módulo} ("tamanho") de um número complexo é definido por

$$\mid z \mid = \sqrt{x^2 + y^2}$$
\end{definition}

\begin{definition}[O Conjugado]
Seja $z = x + iy$ um complexo, então o \textbf{conjurado} de um número complexo é definido por

$$\overline{z} = x - i y$$
\end{definition}

\begin{prop}[Propriedades do conjugado] As seguintes propriedades valem para quaisquer $z, w \in \mathbb{C}$:

\begin{enumerate}[label=(\alph*)]
    \item $\overline{\overline{z}} = z$, $\overline{z \pm w} = \overline{z} \pm \overline{w}$ e $\overline{z w} = \overline{z} \overline{w}$
    
    \item $\overline{z/w} = \overline{z}/\overline{w}$ se $w \neq 0$
    
    \item $z + \overline{z} = 2 Re(z)$ e $z - \overline{z} = 2i Img(z)$
    
    \item $z \in \mathbb{R}$ se e somente se $\overline{z} = z$
    
    \item z é imaginário puro se e somente se $\overline{z} = z$
\end{enumerate}
\end{prop}

\begin{definition}[A Forma Polar]
Seja $z = x + iy$ com $z \neq 0$, então podemos escrever z como

$$z = r(cos(\theta) + sen(\theta))$$

Com as seguintes propriedades

\begin{enumerate}
    \item $r = \mid z \mid $
    
    \item $cos(\theta) = \frac{x}{\mid r \mid}$
    
    \item $sen(\theta) = \frac{y}{\mid x \mid}$
\end{enumerate}
\end{definition}

\begin{theorem}
Seja $n \in \mathbb{Z}_{++}$ e $z = r(cos(\theta) + i sen(\theta))$. Então

$$z^n = r^n (cos(n\theta) + i sen(n\theta))$$
\end{theorem}

\section*{Exponencial, Limite e Derivada (Week 2)}
\label{s2}
\begin{definition}[Função exponencial]
Seja $z \in \mathbb{C}$ com $z = x + iy,\ x,y \in \mathbb{R}$, então 

$$e^z := e^x(cos(y) + i sen(y))$$
\end{definition}

\begin{definition}[Cosseno e seno complexo]
Para $z \in \mathbb{C}$, vamos definir

$$cos(z) = \frac{1}{2}(e^{iz} + e^{- iz})$$

$$sen(z) = \frac{1}{2 i}(e^{- iz} - e^{- iz})$$
\end{definition}


\begin{prop}[Cos e sen]
Seja $z = x + iz,\ x, y \in \mathbb{R}$. Então

\begin{enumerate}[label=(\alph*)]
    \item $cos(z) = cos(x) cosh(y) - i sen(x) senh(y)$
    
    \item $sen(z) = sen(x) cosh(y) + i cos(x) senh(y)$
    
    \item $\mid cos z \mid^2 = cos^2(x) + senh^2(y)$
    
    \item $\mid sen z \mid^2 = sen^2(x) + senh^2(y)$
\end{enumerate}
\end{prop}

\begin{definition}[Função logaritmo]
Seja $z \in \mathbb{C},\ z \neq 0$\footnote{Aqui: $Arg(z) = \theta,\ \theta \in (-\pi,\pi]$ e $Arg(z) = \theta$}

$$Ln(z) = ln \mid z \mid + i Arg(z)$$

$$ln(z) = ln \mid z \mid + i arg(z)$$
\end{definition}

\begin{definition}[Limite]
Seja $z_0 \in \mathbb{C}$ um ponto de acumulação de $D \subset \mathbb{C}$ e seja $f: D \rightarrow \mathbb{C}$. Dizemos que

$$lim_{z \rightarrow z_0} f(z) = l$$

Quando para todo $\varepsilon > 0$, $\exists \delta > 0$ tal que 

$$z \in D - \{z_0\} \text{ e } \mid z - z_0 \mid < \delta \Longrightarrow{} \mid f(z) - l \mid < \varepsilon$$
\end{definition}

\begin{definition}[Continuidade]
Seja $f: D \subset \mathbb{C} \rightarrow \mathbb{C}$ e $z_0 \in D$. Dizemos que f é \textbf{contínua} em $z_0$ se para todo $\varepsilon > 0$, $\exists \delta > 0$ tal que

$$z \in D - \{z_0\} \text{ e } \mid z - z_0 \mid < \delta \Longrightarrow{} \mid f(z) - f(z_0) \mid < \varepsilon$$
\end{definition}

\begin{definition}[Diferencíabilidade]
Seja $f: D \subset \mathbb{C} \rightarrow \mathbb{C}$ e $z_0 \in D$ ponto de acumulação de D. Se existe o limite

$$lim_{z \rightarrow z_0} \frac{f(z) - f(z_0)}{z - z_0}$$

dizemos que f é diferenciável em $z_0$ (ou derivável) e denotamos o limite acima por $f'(z_0)$.
\end{definition}

\begin{definition}[Funções Analiticas]
Seja $f: D \subset \mathbb{C} \rightarrow \mathbb{C}$, f é dita \textbf{analítica} no domínio D se f é diferenciável em todos os pontos de D. E também é dita \textbf{analítica em um ponto} $z_0 \in D$ se f é analítica em uma vizinhança de $z_0$.
\end{definition}


%\section*{... (Week 3)}
%\label{s3}


\end{document}
